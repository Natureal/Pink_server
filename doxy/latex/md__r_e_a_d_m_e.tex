{\bfseries Navigator}\+: {\bfseries https\+://github.com/\+Natureal/\+Pink\+\_\+server/blob/master/knowledge/architecture.\+md \char`\"{}架构与模块\char`\"{}} $\vert$$\vert$ {\bfseries https\+://github.com/\+Natureal/\+Pink\+\_\+server/blob/master/knowledge/evaluation.\+md \char`\"{}压力测试\char`\"{}} $\vert$$\vert$ {\bfseries https\+://github.com/\+Natureal/\+Pink\+\_\+server/blob/master/knowledge/problems.\+md \char`\"{}问题记录\char`\"{}} $\vert$$\vert$ {\bfseries https\+://github.com/\+Natureal/\+Pink\+\_\+server/blob/master/knowledge/basic.\+md \char`\"{}基础知识\char`\"{}}

{\bfseries Motivation}\+: 翻阅了书本以及学习了一些开源项目后,决定自己写一个 server,串联知识,探索一些小想法。 



\subsubsection*{Tools}


\begin{DoxyItemize}
\item {\bfseries 1. 开发环境}
\end{DoxyItemize}

(1) Ubuntu 18.\+04 (Linux Core 4.\+15)

(2) Intel i5-\/8250U (1.\+6(max\+: 3.\+4)G\+Hz $\ast$ 8)


\begin{DoxyItemize}
\item {\bfseries 2. 开发工具}
\end{DoxyItemize}

(1) Vim + Sublime Text 3

(2) C\+Make + ctags (源码定位器)

(3) doxygen (强大的调用关系图工具) \href{https://blog.csdn.net/ZeroLiko/article/details/78162408}{\tt Reference}


\begin{DoxyItemize}
\item {\bfseries 3. 测压工具}
\end{DoxyItemize}

(1) 单线程 I/O 复用方式\+: ./test/pressure\+\_\+test.cpp

(2) 多进程并发方法\+: {\bfseries \href{http://home.tiscali.cz/~cz210552/webbench.html}{\tt webbench}}

(3) 多线程 I/O 复用方式 (最高压力)\+: {\bfseries \href{https://github.com/wg/wrk}{\tt wrk}} 



\subsubsection*{Configuration}


\begin{DoxyItemize}
\item 端口号\+: 7777
\item 页面根目录\+: ./web
\item 连接池初始连接数\+: 2000(最大\+: 10\+K)
\item 线程池线程数量\+: 4
\item 非活动连接超时时间\+: 10s
\item 定期处理超时事件\+: 5s
\item Epoll 模式\+: E\+T/\+LT 


\end{DoxyItemize}

\subsubsection*{To do list\+:}


\begin{DoxyEnumerate}
\item 实现自销毁的线程池(智能指针) {\bfseries Finished v1.\+0}
\item 添加定时器,实现自销毁的时间堆 {\bfseries Finished v1.\+0}
\item 优化定时器触发机制,学习内核 hrtimer
\item 实现自销毁的连接池 {\bfseries Finished v1.\+0}
\item 实现内存池(学习 S\+TL memory pool/tcmalloc/jemalloc)$\ast$$\ast$doing$\ast$$\ast$
\item 优化并行模式 -\/$>$ 优化的 Reactor 模式,省略一次性完成的写完成事件注册 {\bfseries Finished v1.\+0}
\item 生产者消费者,降低耦合 {\bfseries Finished v1.\+0}
\item 守护进程配置
\item 在线修改配置参数
\item 实现负载均衡(学习\+Ngin\+X)
\item 考虑https\+://github.com/\+Natureal/\+Pink\+\_\+server/blob/master/knowledge/E6\%83\%8AE7BEA4E9\%97AEE9A2\%98.\+md \char`\"{}线程池惊群问题\char`\"{} {\bfseries doing}
\item 探索 Proactor 模式\+: 完全异步 + 非阻塞模式 (A\+IO)
\item 考虑 pipeline 技术
\item 日志系统 
\end{DoxyEnumerate}