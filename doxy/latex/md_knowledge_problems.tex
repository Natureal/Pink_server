\subsubsection*{1. 指针的引用}

在 \hyperlink{pink__http__machine_8cpp}{pink\+\_\+http\+\_\+machine.\+cpp} 中,传递给工具函数 \hyperlink{pink__http__machine_8cpp_a9a6484ca3021c09b306a1752e8d2f2d7}{check\+\_\+and\+\_\+move()} 的 text 指针必须是引用形式,否则无法在函数内对其本身的值进行更改

\subsubsection*{2. 静态常量类成员的类外声明}

在 \hyperlink{pink__http__machine_8h}{pink\+\_\+http\+\_\+machine.\+h} 中,静态成员 F\+I\+L\+E\+N\+A\+M\+E\+\_\+\+L\+EN 由于声明为常量型,所以可以在类内直接初始化,但是还应该在类外声明以下(此时不用带 static)。

\subsubsection*{3. 类内 enum 的声明}

在 \hyperlink{pink__http__machine_8h}{pink\+\_\+http\+\_\+machine.\+h} 中,类内写的 enum 只是一种声明,并非定义,所以它们不占内存。

\subsubsection*{4. 使用 using std\+::xxx 简化代码}

在 \hyperlink{main_8cpp}{main.\+cpp} 以及其他代码中,可以考虑多使用 using 来简化代码

\subsubsection*{5. 使用 swap 来释放 vector 所占内存}

在 \hyperlink{main_8cpp}{main.\+cpp} 中,用 swap 来释放 user 指向的 vector,并且在交换后记得令 user=nullptr,防止出现野指针。

\subsubsection*{6. switch 的判断变量}

只能用表达式,整形/布尔型/字符型/枚举型。

\subsubsection*{7. size\+\_\+t}

其类型为 unsigned int / unsigned long

\subsubsection*{8. goto 语句}

goto 有一个非常致命的缺点:不能跳过变量初始化。否则有\+:crosses initialization of ... 错误。

\subsubsection*{9. perror}

作用\+:将上一个函数发生的错误原因,输出到标准设备。

\subsubsection*{10. extern 变量的初始化}

在变量初始化的文件中不要用 extern 了,否则会报错(或者警告)。

\subsubsection*{11. webbench 的请求报文格式}

由于 H\+T\+T\+P/1.\+1 普遍采用 Host 字段,所以原来的 U\+RL 只是即相对资源路径。~\newline
 Host 是 H\+T\+T\+P/1.\+1 特有的,具体和 1.\+0 的区别可以参考:https\+://www.cnblogs.\+com/sue7/p/9414311.html

\subsubsection*{12. webbench 中的 write 和 read 为阻塞}

在第一次测试的时候 webbench 一直卡住,查了源码发现是其用了 read 阻塞读。

\subsubsection*{13. 关闭连接时忘记 close(fd)}

\subsubsection*{14. E\+P\+O\+L\+L\+R\+D\+H\+UP 不可靠!}

在给监听 socket 开启 E\+P\+O\+L\+L\+R\+D\+H\+UP 后,客户端一连接上并发数据,这个错误就跳出来。 E\+P\+O\+L\+L\+R\+D\+H\+UP indeed comes if you continue to poll after receiving a zero-\/byte read.

参考:https\+://stackoverflow.com/questions/27175281/epollrdhup-\/not-\/reliable

\subsubsection*{15. recv(, , 0) 返回 0,在尚未读入数据的情况下}

解决方法:给 recv() 的最后一个参数设置成 M\+S\+G\+\_\+\+W\+A\+I\+T\+A\+LL

\subsubsection*{16. epoll 中监听套接字的触发模式很重要}

E\+T/\+LT 的区别

\subsubsection*{17. shared\+\_\+ptr 操作数组的困难}

参考1:https\+://www.cnblogs.\+com/xiaoshiwang/p/9726511.html

数组的智能指针的限制\+: 1,unique\+\_\+ptr的数组智能指针,没有$\ast$和-\/$>$操作,但支持下标操作\mbox{[}\mbox{]} 2,shared\+\_\+ptr的数组智能指针,有$\ast$和-\/$>$操作,但不支持下标操作\mbox{[}\mbox{]},只能通过get()去访问数组的元素。 3,shared\+\_\+ptr的数组智能指针,必须要自定义deleter

\subsubsection*{18. 静态函数指针的初始化}

在 \hyperlink{classpink__http__conn}{pink\+\_\+http\+\_\+conn} 中设置的静态回调函数(用于放回连接到连接池)指针需要在 cpp 中进行初始化。

\subsubsection*{19. 连接池链表需要线程安全}

通过互斥锁实现连接体链表的线程安全

\subsubsection*{20. 如果将时间堆分离出一个头文件和cpp会出现 undefined referrence to 链接问题(待解决)}

\subsubsection*{21. unique\+\_\+ptr 的初始化方式与 shared\+\_\+ptr 不同}

可以先赋值为 nullptr,再用 unique\+\_\+ptr.\+reset(new object) 的形式

\subsubsection*{22. 工作线程逻辑的优化}

在解析完 H\+T\+TP 请求报文后尝试写出响应报文,只有当该次非阻塞 write 返回 E\+A\+G\+A\+IN 才注册 E\+P\+O\+L\+L\+O\+UT 事件,以提高效率。

\subsubsection*{23. 关闭连接体的操作应该采用统一的接口}

全部统一到 \hyperlink{classpink__http__conn_abdcd7c0da8072d62cb7212523f20298b}{pink\+\_\+http\+\_\+conn.\+close\+\_\+conn()} 函数中。

\subsubsection*{24.}